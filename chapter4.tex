\chapter{Face Tracking}
%Within the significant toolbox of mathematical tools that can be used for stochastic estimation from noisy sensor measurements, one of the most well-known and often-used tools is what is known as the Kalman filter. The Kalman filter is named after Rudolph E. Kalman, who in 1960 published his famous paper describing a recursive solution to the discrete-data linear filtering problem (Kalman 1960).

%The Kalman filter is essentially a set of mathematical equations that implement a predictor-corrector type estimator that is optimal in the sense that it minimizes the estimated error covariance—when some presumed conditions are met. Since the time of its introduction, the Kalman filter has been the subject of extensive research and application, particularly in the area of autonomous or assisted navigation. This is likely due in large part to advances in digital computing that made the use of the filter practical, but also to the relative simplicity and robust nature of the filter itself. Rarely do the conditions necessary for optimality actually exist, and yet the filter apparently works well for many applications in spite of this situation.

%The Kalman filter is an estimator for what is called the linear-quadratic problem, which is the problem of estimating the instantaneous "state" of a linear dynamic system perturbed by white noise - by using measurements linearly related to the state but corrupted by white noise.

%The algorithm works in a two-step process. In the prediction step, the Kalman filter produces estimates of the current state variables, along with their uncertainties. Once the outcome of the next measurement (necessarily corrupted with some amount of error, including random noise) is observed, these estimates are updated using a weighted average, with more weight being given to estimates with higher certainty. The algorithm is recursive. It can run in real time, using only the present input measurements and the previously calculated state and its uncertainty matrix; no additional past information is required.

\section{Procedure} 




