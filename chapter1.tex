\chapter{Introduction}

%\begin{figure}[h]
%\psfrag{A}{$d^2$}
%\includegraphics[scale=0.5]{images/drawing.eps}
%\end{figure}


This is a textual citation \citet{shannon44}. And this is a parenthetical citation \citep{shannon44}. You probably want to use the latter more often.

\section{Moving On}
Let's demonstrate a figure by looking at Fig.~\ref{bandwidth}. 

\begin{figure}[!h]
% Use "\centering" in floats (figure, table), but if you need to center
% some text (why?) use "\begin{center}...\end{center}".
\centering 
% Figure environments same as 0.8 * \textwidth please
% That does not necessarily mean the actual picture size,
% it is a guideline for the environment which could contain
% 2 or more pictures! Be consistent and follow the guidelines
% provided in your sources.
\includegraphics[width=0.8\textwidth]{images/bandwidth-colour.png}
\caption{Planning community bandwidth sharing costs. 
  Note caption capitalization.}
\label{bandwidth} 
% if you move the label it breaks the reference numbering; 
% always have it *after* the caption.
\end{figure}

Remember how to include code with {\tt verbatim} 
and to fix the tabs in {\sf python} in a verbatim environment? 
It may be best to have an `include' command for code, 
not to have to re-edit it all the time.
\verbatimtabinput{code/mycode.py}


